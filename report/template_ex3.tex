\documentclass[a4paper,12pt,oneside,onecolumn]{article} % Document type

\usepackage[left=1.0in, right=1.0in, top=1.0in, bottom=1.0in]{geometry}

\usepackage{fontspec}
\setmainfont{cmr12}
\usepackage{amsmath,amssymb}   % Contains mathematical symbols
\usepackage{xltxtra}
\defaultfontfeatures{Ligatures=TeX}
\usepackage[english]{babel}    % Language
\usepackage[square,numbers]{natbib}     %Nice numbered citations
\bibliographystyle{plainnat}            %Sorted bibliography
\usepackage{float}
\usepackage{pgfplots}
\usepackage{tikz}
\usepackage{lscape}
\usepackage{graphicx}
\usepackage{pstricks}
\usepackage{subfigure}
\usepackage{multicol}
\usepackage{caption}
\usepackage{rotating}
\usepackage{lastpage}
\usepackage{slantsc}
\usepackage{lmodern}
\usepackage{fancyhdr}
\pagestyle{fancy}
\lhead{Filotheou}
\rhead{EL2520}
\chead{H-inf control, Robustness}
\rfoot{Page \thepage\ of \pageref{LastPage}}
\cfoot{}
\renewcommand{\footrulewidth}{0.4pt}

\title{
	Computer Exercise 3\\
	EL2520 Control Theory and Practice
}
\author{
  Alexandros Filotheou \\
  alefil@kth.se \\
  871108-5590
}

\newcommand{\imageg}[3][width=1.0\columnwidth]{
	\begin{figure}[H]
		\centering
	    \includegraphics[#1]{#2}
		\caption{#3}
		\label{fig:#2}
	\end{figure}
}

\newcommand{\imaget}[3][width=1.0\columnwidth]{
	\begin{figure}[H]
		\centering
      \input{#2}
		\caption{#3}
		\label{fig:#2}
	\end{figure}
}

\begin{document}
	\maketitle

	% Suppression of disturbances
	\section*{Suppression of disturbances}

  The process to be controlled is identified by the transfer function $G$:

  \begin{align*}
    G(s) &= 10^4 \dfrac{s+2}{(s+3)(s+100)^2}
  \end{align*}
	The weight $W_S$ is
	\begin{align*}
    W_S(s) &= \dfrac{10^4}{(s+p_1)(s+p_2)}  \\
    p_k &= -0.1 \pm i\sqrt{(100\pi)^2 - 0.1^2}, k = \{1,2\}
	\end{align*}

  Figure \ref{fig:figures/macro_1.pdf} illustrates the amplitudes of the reference
  signal ($0.0$), the output, the control signal ($10.0$) and the disturbance
  ($1.0$). The amplitude of the output is at most $3\cdot10^{-5}$
  in steady-state, hence the rate between the disturbance and the output
  oscillations is approximately $3\cdot10^5$.

  If the controller was a proportional one, it would need to have a value around
  $10^8$ for the disturbances to be damped equally well as in the above case.
  However, in this case the amplitude of the control signal would increase by
  $2$ orders of magnitude. At the same time, a P-controller would treat all
  frequencies in the same manner, i.e. it would attenuate all frequencies
  equally; not just the $50$ Hz one.

  \imageg{figures/macro_1.pdf}{Simulation results with system $G$, using $W_S$.}

  \newpage

	% Robustness
	\section*{Robustness}

  The system's transfer function is now given by $G_0$:

  \begin{align*}
    G_0(s) &= G(s)(1+\Delta_G(s)) = 10^4 \dfrac{s+2}{(s+3)(s+100)^2} \cdot \dfrac{s-1}{s+2}
  \end{align*}

  Here, $\Delta_G(s) = -\dfrac{3}{s+2}$, which is stable. According to the small
  gain theorem, the closed loop is stable if

  \begin{align*}
    |T(i\omega)\cdot\Delta_G(i\omega)| &< 1 \Leftrightarrow  \\
    |T(i\omega)| &< \Big|\dfrac{i\omega+2}{3}\Big|
  \end{align*}

  Hence the weight $W_T$ can be chosen as $-\Delta_G$. The weights are then:
	\begin{align*}
    W_S(s) &= \dfrac{10^4}{(s+p_1)(s+p_2)} \\
    W_T(s) &= \dfrac{3}{s+2}
	\end{align*}
  where $p_k = -0.1 \pm i\sqrt{(100\pi)^2 - 0.1^2}, k = \{1,2\}$.  Figure
  \ref{fig:figures/small_gain_2.tex} illustrates that the small gain theorem
  holds: the amplitude of $T\cdot\Delta_G$ is less than one for all frequencies.
  Figure \ref{fig:figures/macro_2.pdf} illustrates the amplitudes of the
  reference signal ($0.0$), the output, the control signal ($\approx10.0$) and
  the disturbance ($1.0$). The amplitude of the output is less than
  $5\cdot10^{-4}$ in steady-state, hence the rate between the disturbance and
  the output oscillations is approximately $2000$, which is less than in the
  previous case.


  \imaget{figures/small_gain_2.tex}{Bode diagram of $T(i\omega)\cdot\Delta_G(i\omega)$. The
    amplitude is always less than 1, hence the small gain theorem is fulfilled.}

	\imageg{figures/macro_2.pdf}{Simulation results with system $G_0$, using $W_S$ and $W_T$.}

  \newpage

	% Control signal
	\section*{Control signal}

	The weights are
	\begin{align*}
    W_S(s) &= \dfrac{100}{(s+p_1)(s+p_2)} \\
    W_T(s) &= \dfrac{3}{s+2}              \\
		W_U(s) &= \dfrac{100}{(s+p_1)(s+p_2)}
	\end{align*}
  where $p_k = -0.1 \pm i\sqrt{(100\pi)^2 - 0.1^2}, k = \{1,2\}$.

	\imageg{figures/macro_3.pdf}{Simulation results with system $G_0$, using $W_S$, $W_T$ and $W_U$.}

  We can see that a specification on the control signal $u$ to be halved results
  in inadmissible output signal: its value instead of following the reference
  signal shoots up to an amplitude of $0.5$.

\end{document}
